El presente trabajo se centra en el diseño, desarrollo e implementación del Módulo de Ruteo Seguro, el mismo que funcionará como un microservicio autónomo dentro de la arquitectura de Alertify.
En donde el esfuerzo principal se centra en la creación de un Motor de Ruteo que debe ser capaz de consumir datos geográficos de peligrosidad provenientes del Módulo de Reportes, de manera que a través de un algoritmo se pondere la minimización del riesgo sobre la distancia, siendo este el entregable funcional central. A su vez, la implementación de la Base de Datos de Red Vial con capacidad para almacenar y gestionar eficientemente las estructuras de datos espaciales (nodos y caminos) requeridas por el motor de cálculo asegurando así una comunicación eficiente entre el aplicativo móvil, dentro del mismo efecto se desarrollará la API y respectivo frontend los mismos que exprondran la operatividad del motor de ruteo, asegurando que su puesta en marcha cumpla con las correspondientes pruebas funcionales y no funcionales.




%\lipsum[1-3]

%%%---------- Tabla 1. Plan de Tesis ----------%%%
%\input{03Tables/01Chapter/C1.tableExample.tex}